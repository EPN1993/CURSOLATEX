\documentclass[11pt]{report}
\usepackage[utf8]{inputenc}
\usepackage[T1]{fontenc}
\usepackage[spanish , es-nolayout , es-nodecimaldot , es-tabla]{babel}
\usepackage{amsmath , amsfonts , amssymb , amsthm}
\usepackage{enumerate}
\usepackage{enumitem}
\usepackage{parskip}
\usepackage{nicefrac}

\title{TAREA 2}
\author{Gabriela Cando}
\date{\today}

\begin{document}
\maketitle

La optimización de funciones no es un tema analizado únicamente con herramientas del cálculo en
una variable y de la programación lineal. Esta se puede generalizar a espacios más generales como
son los espacios de Banach. A continuación se presenta el siguiente problema de optimización:

\[
 \int_{0}^{a} (u(x))^2 dx + \int_{0}^{a} y(x)^2 dx + \dfrac{a^2}{med(0,a,a^2)}, 
\]

Sujeta a 

\[
\begin{cases}
-u(x) + \alpha(x)u(x)= y & \text{en}  {(0,a)} \\
u=0 & \text{en}  {(0,a)} \\
\lim_{x \to 0} y(x) = a \\
a \geq 4
\end{cases}
\]

\[
 \text{La idea es optimizar sobre el conjunto de funciones de cada intervalo de la forma} \quad [0,a] \qquad \text{y determinar el valor de}  \quad a \geq 4  \qquad \text{que indique el mejor intervalo de trabajo.} 
\]

\end{document}